\documentclass[11pt]{article}
\usepackage[margin=1in]{geometry} 
\usepackage{amsmath,amsthm,amssymb,amsfonts}
\usepackage{amsfonts}

\newcommand{\overbar}[1]{\mkern 1.5mu\overline{\mkern-1.5mu#1\mkern-1.5mu}\mkern 1.5mu}
\newcommand{\abs}[1]{\lvert #1 \rvert}
\newcommand\floor[1]{\lfloor#1\rfloor}
 
 
 
\newenvironment{problem}[2][Problem]{\begin{trivlist}
\item[\hskip \labelsep {\bfseries #1}\hskip \labelsep {\bfseries #2.}]}{\end{trivlist}}
%If you want to title your bold things something different just make another thing exactly like this but replace "problem" with the name of the thing you want, like theorem or lemma or whatever
 
\begin{document}
\title{Assignment 2}
\author{Ao Zeng}
\maketitle
\begin{problem}{1}
\textit{(a)} As $G = \mathbb{Z} / 12 = \{ 0,1,2,3,4,5,6,7,8,9,10,11 \}$, therefore, $\abs{G} = 12$\\
$\abs{0} = 1, \abs{1} = 12, \abs{2} = 6, \abs{3} = 4, \abs{6} = 2, \abs{7} =12 , \abs{8} = 4, \abs{9} = 4, \abs{10} = 6, \abs{11}=12$\\
\textit{(b)} As $G = (\mathbb{Z}/12)^{\times} = \{ 1,5,7,11\}$, therefore, $\abs{G} = 4$\\
$\abs{1} = 1, \abs{5} = 2, \abs{7} = 2, \abs{11} = 2$\\
\textit{(c)} As $G = (\mathbb{Z}/16)^{\times} = \{ 1, 3, 5, 7, 9, 11, 13, 15 \}$, therefore $\abs{G} = 8$\\
$\abs{1} = 1, \abs{3} = 4, \abs{5} = 4, \abs{7} = 2, \abs{9} = 2, \abs{11} = 4, \abs{13} = 4, \abs{15} = 2$\\
\textit{(d)} As $G = \mathbb{S} = \{ I, R_{\pi/2}, R_{\pi}, R_{3\pi/2}, H, V, D, D' \}$, therefore $\abs{G} = 8$\\
$\abs{I} = 1, \abs{R_{\pi/2}} = 4, \abs{\pi} = 2, \abs{3\pi/2} = 4,\abs{H} = 2, \abs{V} = 2, \abs{D} = 2, \abs{D'} = 2$\\
\end{problem}

\begin{problem}{2}
\textit{(a)}  Claim: if $n$ is not prime, and $n\neq 4$, then $(n-1)! = 0 \mod n$.\\
\textit{proof:} Given $n$ is not a prime number,and $n\neq 4$, we can derive that:\\
\textit{(i)} When $\forall p \in \mathbb{P}, n\neq p^2$:\\
 $\exists l,m, 1<l<n, 1<m<n$, such that $ n = l \times m$.\\
Moreover, $l\neq m$, given that $n$ is not a power of any prime bumber.
Hence, $l, m$ both divide $(n-1)!$,\\
which is equivalent to $( n-1!) \mod n = (1\times 2 \times ... \times l \times ... \times m \times .. \times (n-1)) \mod (l \times m) = 0 $\\
\textit{(ii)} When $\exists p \in \mathbb{P}$ such that $n = p^2$, and $p>2$\\
Then $\exists some 1<k<p$, such that $k\times p <(n-1)$.\\
Thus, $( n-1!) \mod n = (1\times 2 \times ... \times p \times ... \times k\times p \times .. \times (n-1)) \mod (p \times p) = 0 $\\
Since $0 \mod n = 0$, it can be concluded from \textit{(i),(ii)} that $(n-1)! = 0 \mod n = 0$. $\square$\\
\\
\textit{(b)} prove the statement that, if $n = 4$, then $(n-1)!\neq -1 \mod n$\\
As $n = 4$, $(4-1)! \mod 4 = 2$. Since $2\neq (4-1)$, the above statement is proved $\square$.\\
\end{problem}

\begin{problem}{3} Proof by contradiction:\\
Given $a$ being the only element in $G$ of order $2$. Assume that $a \notin \mathbb{Z}(G)$.\\
Then $\exists x \in G$, such that $x\cdot a \neq a \cdot x$\\
However, as $(x\cdot a \cdot x^{-1})^2 = x \cdot a \cdot x^{-1} \cdot x \cdot a \cdot x^{-1} = x\cdot a \cdot e \cdot a \cdot x^{-1} = x \cdot e \cdot x^{-1} = e$,\\
$x\cdot a \cdot x^{-1} = e$ or $x\cdot a \cdot x^{-1} = a$.\\
\textit{(i)} If $x \cdot a \cdot x^{-1}=e$, then $(x\cdot a) =x $, then $a = e$\\
\textit{(ii)} If $x \cdot a \cdot x^{-1}=a$, then $ x \cdot a \cdot x^{-1} \cdot x = a \cdot x$, and it is equivalent to $x\cdot a = a \cdot x$.\\
Since both of these cases \textit{(i)} and \textit{(ii)} are contradictory to our assumpution, therefore, by mathematical contradiction, it can be concluded that $a \in \mathbb{Z}(G)$. $\square$\\
\end{problem}
\newpage

\begin{problem}{4}
Given $H<G, K<G$, and assume that $G = K \cup G$.\\
\textit{(i)}If $H = K$, then $K \cup H  = H < G$, hence $H \neq K$.\\
\textit{(ii)}If $H<K$, then $K \cup H = K <G$, hence $H$ can not be a proper subset of $k$. (Same reasoning for $K<H$ is impossible).\\
\textit{(iii)}If $\exists h \in H, k\in K$ such that $h\notin K$, and $k \notin H$, then since $H<G, K<G$, $hk \in G$.\\
However $hk \notin H$. If $hk \in H$, then $k = h^{-1} hk \in H$ which is contradictory to the assumpution.\\
Also $hk \notin K$, otherwise $h = hkk^{-1}$ would be in $K$, which is also contradictory to the assumption.\\
Therefore, since $hk \notin K$ and $hk \notin H$, $hk \notin K\cup H$, which is equivalent to $G \neq K\cup G$.\\
Hence, it can be concluded form \textit{(i),(ii),(iii)} that the original statement is true.$\square$\\

\end{problem}

\begin{problem}{5}
According to the given condition, we know $a\cdot a = e, b\cdot b =e$, and such $H$ can be found as $H = \{ a, b, e, ab \}$\\
Where $a,b,e,ab$ are all in $G$, and since:\\
\textit{(i)} $H$ is closed under the same binary operation, as $a \cdot a = e$,$a\cdot b = ab$, $a \cdot e = a$, $a\cdot ab = b$;\\
$b\cdot a = ba = ab$, $b\cdot b = e$, $b\cdot e = b$, $b\cdot ab = b\cdot ba = a$;\\
$e \cdot a = a$, $e \cdot b = b$, $e \cdot e = e$, $e \cdot ab = ab$;\\
$ab \cdot a = ba \cdot a = b$, $ ab\cdot b = a$, $ab \cdot e = ab$, $ab\cdot ab = ab\cdot ba = e$\\
\textit{(ii)} $\forall h \in H, h^{-1} \in H$, as $a^{-1} = a \in H$, $b^{-1} = b \in H$, $e^{-1} =e \in H$, $(ab)^{-1} = b^{-1}a^{-1} = ba = ab \in H$\\
\textit{(iii)} $e$ is the same identity in $G$.\\
Therefore, from \textit{(i),(ii),(iii)}, it can be concluded that $H<G$, and $\abs{H}=4$\\
\end{problem}

\end{document}











