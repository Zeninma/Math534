\documentclass[11pt]{article}
\usepackage[margin=1in]{geometry} 
\usepackage{amsmath,amsthm,amssymb,amsfonts}
\usepackage{amsfonts}

\newcommand{\overbar}[1]{\mkern 1.5mu\overline{\mkern-1.5mu#1\mkern-1.5mu}\mkern 1.5mu}
\newcommand{\abs}[1]{\lvert #1 \rvert}
\newcommand\floor[1]{\lfloor#1\rfloor}
 
 
 
\newenvironment{problem}[2][Problem]{\begin{trivlist}
\item[\hskip \labelsep {\bfseries #1}\hskip \labelsep {\bfseries #2.}]}{\end{trivlist}}
%If you want to title your bold things something different just make another thing exactly like this but replace "problem" with the name of the thing you want, like theorem or lemma or whatever
 
\begin{document}
\title{Assignment 2}
\author{Ao Zeng}
\maketitle
\begin{problem}{(1)}
\textit{(a)} Denote $H \cap K = X$.\\
$\forall x,y \in X$, it is obvious that $x,y\in H, h,y\in K$.\\
Moreover, $x\cdot y \in H$ and $K$, due to $H,K$ are subgroups of $G$. Hence $x\cdot y \in X$ as well.\\
Therefore, $\forall x,y \in X, x\cdot \in X$, and this suffices to show $\cdot $ is a binop over $X$.\\
\\
Then since $H, K$ are both subgroups of $G$, $e \in H,G$, therefore $e \in X$.\\
\\
As $H, K$ are both subgroups of $G$, using the same binop, therefore, the operation is still associative
over $X$.\\
\\
Because $X = H \cap K$, $\forall x \in X$, $x \in H,K$ as well. Since $H,K$ are subgroups of $G$, $x^{-1}$ must be in $H, K$.\\
Further implying that $x^{-1} \in X$. Therefore $\forall x\in X, \exists x^{-1} \in X$.
\\
It can be concluded from abve, that $H\cap K \leq G$.\\
\\
\textit{(b)} Assume $\exists b^k \in <b>, \exists a^h \in <a>$ such that $ b^k = a^h$, where $k,h \in \mathbb{Z}_+$.\\
Then $<a> \leq <b>$ or $<b>\leq <a>$. Since $<a>, <b>$ are both finite, hence $\abs{<a>} | \abs{<b>}$ or $\abs{<b>} | \abs{<a>}$.\\
However, as $\abs{<b>}$ and $\abs{<a>}$ are relatively prime, this is a contradiction.\\
Therefore, we can conclude that, $<b> \cap <a> = \{ e \}$.
\end{problem}

\begin{problem}{(2)}
\textit{(a)} \[ \alpha^{-1} =  \begin{pmatrix} 1&2&3&4&5&6 \\ 2&1&3&5&4&6  \end{pmatrix} , \beta^{-1} \begin{pmatrix} 1&2&3&4&5&6 \\ 2&3&5&4&6&1  \end{pmatrix}
\alpha \beta = \begin{pmatrix} 1&2&3&4&5&6 \\ 6&2&1&5&3&4\end{pmatrix}, \beta \alpha = \begin{pmatrix} 1&2&3&4&5&6 \\ 1&6&2&3&4&5 \end{pmatrix}\]\\
\textit{(b)} $\alpha = (1,2)(3)(4,5)(6). \beta = (1,6,5,3,2)(4)$\\
\[ \textit{(c)} \beta^{-1} \alpha \beta = \begin{pmatrix} 1&2&3&4&5&6\\ 1&3&2&6&5&4 \end{pmatrix} = (1)(2,3)(4,6)(5), \alpha^{-1}\beta \alpha = 
\begin{pmatrix} 1&2&3&4&5&6 \\ 2&6&1&3&5&4 \end{pmatrix} = (1,2,6,4,3)(5) \]
\end{problem}
\begin{problem}{(3)}
For elements of order $4$ in $S_6$, the possible combination of cycles lengths can only be $1,1,4$ or $2,4$, due to the order is the least common multiple
of sycles' lengths\\
When the cycles lengths are $1,1,4$, there are $1\times 1 \times 4! = 4!$ elements. \\
For the cycles lengths are $2,4$, there are $2! \times 4! = 2 \times 4!$ elements.\\
Therefore the total number of elements of order $4$ is $4!+2\times 4! = 3\times 4!$.\\
Similarly for order $2$, the possible combinations are $2,2,2$, $1,1,1,1,2$, $1,1,2,2$, and the total number of elements according to these combinations are:\\
$2!\times 2! \times 2! + 2! + 2!\times 2! = 8+2+4 = 14$.\\
\end{problem}

\begin{problem}{(4)}
\textit{(a)} The out shuffle is a permutation on the deck of card, as it is a bijiective function of the orders of the cards.\\
Then Denote the function as $\sigma, \sigma = (1)(2,3,5,9,17,33,14,27)(4,7,13,25,49,46,40,28)(6,11,21,41,30,8,15,29)(10,19,37,22,43,34,16,31)(12,23,45,38,24,47,42,32)
(18,35)(20,39,26,51,50,48,44,36)(52)$\\
Above, the cycle lengths are $1,8,8,8,8,8,2,8,1$ respectively. Therefore, the oder of $\sigma$ is $lcm (1,2,8) = 8$.\\
Therefore $8$ is the least number of out shuffles to return cards to its original position.\\
\textit{(b)} Let the in shuffles denote by \[ \sigma ' = \begin{pmatrix} 1&2&3&4&5&6&7&8&9&10\\ 2&4&6&8&10&1&3&5&7&9 \end{pmatrix} = (1,2,4,8,5,10,9,7,3,6)\]\\
Since the length of cycle is $10$, at least 10 perfect in shuffles are needed to return the cards back to its original position.\\
\end{problem}

\begin{problem}{(5)}

\end{problem}
\end{document}











