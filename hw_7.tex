\documentclass[11pt]{article}
\usepackage[margin=1in]{geometry} 
\usepackage{amsmath,amsthm,amssymb,amsfonts}
\usepackage{amsfonts}

\newcommand{\overbar}[1]{\mkern 1.5mu\overline{\mkern-1.5mu#1\mkern-1.5mu}\mkern 1.5mu}
\newcommand{\abs}[1]{\lvert #1 \rvert}
\newcommand\floor[1]{\lfloor#1\rfloor}
 
 
 
\newenvironment{problem}[2][Problem]{\begin{trivlist}
\item[\hskip \labelsep {\bfseries #1}\hskip \labelsep {\bfseries #2.}]}{\end{trivlist}}
%If you want to title your bold things something different just make another thing exactly like this but replace "problem" with the name of the thing you want, like theorem or lemma or whatever
 
\begin{document}
\title{Assignment 7}
\author{Ao Zeng}
\maketitle
\begin{problem}{1}
\textit{(a)} Define $\phi : G \rightarrow H$, where $\forall g \in G, \phi (g) = 2g$, and according to definition, $\phi(g) \in H$.\\
First $\phi$ a homomorphism, as $\forall g,h \in G, \phi(g+h) = 2g + 2h = \phi(g)+phi(h)$.\\
Then as $\forall \phi(k_1), \phi(k_2) \in H$, if $\phi(k_1) = \phi(k_2)$, then $2k_1 = 2k_2$, and $k_1 = k_2$, hence $\phi$ is injective.\\
Then since $\forall h \in H, \exists k\in G$ such that $2k = h$, hence $\phi$ is sujective.\\
Therefore, it can be concluded that $\phi: G \rightarrow H$ is a bijective homomorphism, which implies that it is an isomorphism.$\square$\\
\textit{(b)} $\forall z \in \mathbb{Z}$, we can find a subgroup $H_z = \{ zk| \forall k \in G\}$, which G is isomorphic to, by following the same proof above, but substitute $2$ with $z$.\\
As $|\mathbb{Z}|$ is accountably infinite, there in fact infinitely many subgroups of $G$ to which it is isomorphic. $\square$\\
\end{problem}

\begin{problem}{2}
Assume $\exists \phi : \mathbb{R}^{\times} \rightarrow (R,+)$, which is a isomorphism.\\
Then $\forall a \in \mathbb{R}^{\times}, \phi(a \times a) = \phi(a) + \phi(a)$, which implies to that $\phi(a\times a) = 2\phi(a)$.\\
According to the properties of homomorphism, $\phi(a\times a) = (\phi(a))^2$, hence $(\phi(a))^2 = 2\phi(a)$.\\
By solving the equation $\phi(a)$ can only be $0$ or $2$.\\
However, since $a$ is any number in $\mathbb{R}^{\times}$, therefore, $\phi$ is not one-to-one, which is agianst our assumption.\\
Therefore, this group is not isomorphic to $(\mathbb{R},+)$. $\square$\\
\end{problem}

\begin{problem}{3}
\textit{(i)} $\mathbb{Z}/15 = \{0,1,2,3,4,5,6,7,8,9,10,11,12,13,14 \}$.\\
Let $\phi: \mathbb{Z}/15 \rightarrow <(1,2,3)(4,5,6,7,8)>$, where $ \phi(n) = ((1,2,3)(4,5,6,7,8))^n$.\\
Then $\forall x,y \in \mathbb{Z}/15, \phi(x+y) = ((1,2,3)(4,5,6,7,8))^(x+y)=\phi(x) \phi(y)$, hence $\phi$ is a homomorphism.\\
Also, if $((1,2,3)(4,5,6,7,8))^n = ((1,2,3)(4,5,6,7,8))^m, m,n = 0,1,2,...14$, then $m = n$, which implies that $\phi$ is injective.\\
And, $\abs{ \mathbb{Z}/15 } = 8 = \abs{<(1,2,3)(4,5,6,7,8)>}$, by the fact that it is already injective $\phi$ is also sujective, and hence bijective.\\
Hence $\phi$ is an isomorphism between $\mathbb{Z}/15$ and $<(1,2,3)(4,5,6,7,8)>$.\\
Moreover, since $(1,2,3)(4,5,6,7,8) \in S_8, <(1,2,3)(4,5,6,7,8)> \leq S_8$ as well.\\
Therefore, $S_8$ contains subgroups isomorphic to $\mathbb{Z}/15$.\\
\textit{(ii)} According to \textit{Cayley's Thm}, $\exists$ an isomorphism $\phi: (\mathbb{Z}/16)^{\times} \rightarrow \phi((\mathbb{Z}/16)^{\times})$,
where $\phi((\mathbb{Z}/16)^{\times}) \leq S_{(\mathbb{Z}/16)^{\times}}$.\\
As $\abs{(\mathbb{Z}/16)^{\times} }= 8$, therefore, $\exists$ an isomorphism $\phi: (\mathbb{Z}/16)^{\times} \rightarrow \phi((\mathbb{Z}/16)^{\times})$,
where $\phi((\mathbb{Z}/16)^{\times}) \leq S_{8}$, according to the proposition that is covered in the class.\\
Therefore it can be concluded that $S_8$ contains subgroups isomorphic to $(\mathbb{Z}/16)^{\times}$.\\
\textit{(iii)} $D_8 = {I,R_{45}, R_{90}, R_{135},...,R_{335},F_1,F_2,F_3,F_4}$, by the proposition covered in class, we can find an isomorphism $\phi$, which
is defined as below:\\
$\phi(I) = e$, $\phi(R_{45 \times n}) = (1,2,3,4,5,6,7,8)^n$, $\phi(F_1) = (2,8)(3,7)(4,6)$, $\phi(F_2) = (1,3)(4,8)(5,7)$,$\phi(F_3) = (1,5)(2,4)(6,8)$,$\phi(F_4) = (1,7)(2,6)(3,5)$\\
Which by the same proposition, $\phi$ is an isomorphism from $D_n$ to a subgroup of $S_n$\\
\end{problem}

\begin{problem}{4}
\textit{(i)} Fixed $a$, $\forall x \in a(H\cap K)$, $\exits y \in H\cap K$ such that $x = ay$.\\
Then $y \in H and y\in K$, thus $x\in aH, x\in aK$, which implies that $x\in (aH)\cap (aK)$.\\
Therefore $a(H \cap K) \subset (aH)\cap(aK)$.\\
\textit{(ii)} $\for x \in (aH)\cap (aK), \exists K \in K, h \in H$ such that $x = ah = ak$.\\
Let $ y = h = k$, then $y \in (H\cap K)$, which implies that $x = ay \in a (H \cap K)$.\\
Therefore $ (aH)\cap(aK) \subset a(H \cap K)$.\\
Therefore, from \textit{(i),(ii)}, it can be concluded that $ (aH)\cap(aK) = a(H \cap K)$. $\square$\\
\end{problem}

\begin{problem}{5}
$\forall n > 2, n -1 \in \abs{ (\mathbb{Z} /n)^{\times} }$.\\
Moreover, $\abs{n-1} = 2$, as $(n-1)^2 mod n = (n^2 -2n +1) mod n = n^2 mod n - 2n mod n +1 mod n =1$, which is also the identity.\\
Therefore, $\forall n > 2, \exists x \in (\mathbb{Z} /n)^{\times}$ such that $\abs{x} = 2$.\\
According to \textit{Theorem of Lagrange}, $2 | (\mathbb{Z} /n)^{\times}$, which is equivalent to it is always even. $\square$\\
\end{problem}
\end{document}











