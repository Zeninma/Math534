\documentclass[11pt]{article}
\usepackage[margin=1in]{geometry} 
\usepackage{amsmath,amsthm,amssymb,amsfonts}
\usepackage{amsfonts}

\newcommand{\overbar}[1]{\mkern 1.5mu\overline{\mkern-1.5mu#1\mkern-1.5mu}\mkern 1.5mu}
\newcommand{\abs}[1]{\lvert #1 \rvert}
\newcommand\floor[1]{\lfloor#1\rfloor}
 
 
 
\newenvironment{problem}[2][Problem]{\begin{trivlist}
\item[\hskip \labelsep {\bfseries #1}\hskip \labelsep {\bfseries #2.}]}{\end{trivlist}}
%If you want to title your bold things something different just make another thing exactly like this but replace "problem" with the name of the thing you want, like theorem or lemma or whatever
 
\begin{document}
\title{Assignment 9}
\author{Ao Zeng}
\maketitle
\begin{problem}{1}
\textit{(a)}$\forall g \in G, gcd(g,200) =1$, therefore, $G \subset (\mathbb{Z}/200)^{\times}$\\
Let $ e = 1$, $\forall g \in G, eg = g = ge$, therefore $G$ has identity.\\
ince $(\mathbb{Z}/200)^{\times}$ is an abelian group $\froall g_1,g_2 \in G, g_1 g_2 = g_2 g_1$.\\
Therefore, given $g$, $\forall g' in G$, if $g g' \in G$, for every $g \in G$. It suffices to prove $G$ is closed, under the same binary operation.\\
According to the table attached to the homework,$G$ is closed under the given binary operation.\\
Therefore it can be concluded that $G \leq (\mathbb{Z}/200)^{\times}$. $\square$\\
\textit{(b)}\\
Cyclic groups generated by each elements in $G$ are $\{1\},\{1, 7, 49, 143\}, \{1, 43, 49, 107\},  \{1,49\}, \{ 1, 51 \}$\\
$\{1, 57, 49, 193\}, \{ 1, 93, 49, 157\}, \{ 1, 99\}, \{ 1, 101\}, \{1, 107, 49, 43\},\{1, 143, 49, 7\},\{ 1, 149\}, \{1, 151\}$\\
$\{1, 157,49,  93\}, \{1, 193 ,49 , 57\}. \{1,199\}$.\\
Hence in $G$, there are $1$ element with order $1$, $7$ elements with order $2$ ,$8$ elements with order $4$, as shown above.\\
Hence, $G$ is isomorphic to the group $H = \mathbb{Z}/2 \times \mathbb{Z}/2 \times \mathbb{Z}/4$.\\
Since $H$ has one element with order $1$, that is ${0,0,0}$, and $2\times 2 \times 2 = 8$ elements of order $4$, and $16-1-8 = 7$ elements with order $2$.\\
\end{problem}

\begin{problem}{2}
According to \textit{Fundamental Theorem of Finite Abelian Groups}, an the fact that, $\frac{\mathbb{Z}}{n_1} \times \frac{\mathbb{Z}}{n_2} \times... \frac{\mathbb{Z}}{n_r}$ is cyclic \textit{iff} $\gcd{n_i,n_j} = 1, \forall i,j$.\\
Thus, Given positive integer $n$, in order to let every abelian group of order $n$ is cyclic, $n = p_1 \times p_2 \times ... \times p_n$, where $p_i$ is prime for all $i$, and $\forall i \neq j,p_i \neq p_j$.
\end{problem}

\begin{problem}{3}
\textit{(a)} proof:\\
Since $(-a) \cdot (-b) = a \cdot b$, then $ -((-a) \cdot (-b)) + (a\cdot b) = 0$, where $0$ is also the additive identity.\\
By the fact that $x \cdot (-y) = -(x \cdot y) = (-x) \cdot y$, it can be derived that $(-(-a))\cdot (-b) + (a\cdot b) =0$, which is equivalent to that $a \cdot (-b) + (a\cdot b) = 0$\\.
Then by the \textit{distributive property} of the mulptiplicative operation, it can be derived that,
$a \cdot ((-b) + b) =0$.\\
As $RHS = a \cdot 0 = 0$, and $LHS = 0$, thus $RHS =LHS$, and the original statement had been proved. $\square$\\
\textit{(b)}\\
According to what has been proved in the class, $(-1) \cdot a = 1 \cdot (-a)$, since $1$ is the identity, $1 \cdot (-a) = -a$. Therefore, the statement has been proved$\square$\\
\textit{(c)}\\
Assume $\forall a \in R, \exists b,c \in R$ such that $a \cdot b = b\cdot a = a, a\cdot c = c\cdot a = a$.\\
Then $a\cdot b - a\cdot c = e$, where $e$ is the additive identity.\\
By the \textit{distributive property}, it can be derived that $a\cdot (b - c) = e$, thus $b-c = e$, which implies that $ b= c$.\\
Therefore, identity $1$ is unique $\square$.\\
\textit{(d)}\\
Let $x,y \in R$ such that $a \cdot x = a\cdot y =1$. Then $a\cdot x - a\cdot y =a \cdot (x - y) =1 - 1 =0$.\\
Therefore $(x-y) = 0$, which implies that $x = y$. Therefore, its inverse is unique$\square$.
\textit{(e)}\\
(1). According to definition, $\cdot$ is a binary operation on $R$, and is associative.\\
(2). According to \textit{(c)} and the definition of $(R^{\times}, \cdot)$ has a unique identity.\\
(3). According to \textit{(d)}, $\forall a \in R^{\times}$, there exists a unique $a^{-1}\in R$, such that $a \cdot a^{\times} = a^{\times}\cdot a = 1$.\\
From above, it can be concluded that $(R^{\times},\cdot)$ is a group$\square$.
\end{problem}

\begin{problem}{(4)}
Since $(R,+)$ is a cyclic group, let $<g> = (R,+)$.\\
Then $\forall a,b \in R, \exists x, y$ such that $a = xg, b = yg$. Then $a\cdot b = xg \cdot yg =(xy)g^2 = yg \cdot xg = b\cdot a$.\\
Therefore, $R$ is a commutative ring $\square$.\\
\end{problem}

\begin{problem}{(5)}
\textit{(i)}
First, Denote $Ann_R(r)$ as H, and prove that $H$ is a subgroup of $R^+$ in the following:\\
\textit{(a)} Identity $0 \in H$, since $0 \cdot r =0$.\\
\textit{(b)} $\forall a \in H$, $-a \in H$ as well, since $(-a) \cdot r = -(a \cdot r) = -0 = 0$, as $a\cdot r = 0$\\
\textit{(c)} $\forall a,b \in H$, $a\cdot r = b\cdot r = 0$.\\
Then $(a+b) \in H$ as well, since $(a+b)\cdot r = a\cdot r + b\cdot r = 0 + 0 =0$\\
Therefore, it can be concluded that $H^+$ is a subgroup of $R^+$.\\
Finally, $\forall a,b \in H$, $a\cdot b \in H$ as well, as $(a\cdot b) \cdot r = a\cdot (b\cdot r) = a \cdot 0 = 0$\\
Hence, $H$ is a subring of $R$.\\
\textit{(ii)}
Let $R = (\mathbb{Z}, +, \times)$, let $r = 0$, then $Ann_R(r) = \mathbb{Z}$\\
\end{problem}
\end{document}











